\documentclass{uniquecv}

\usepackage{fontawesome}

% ----------------------------------------------------------------------------- %

\begin{document}

\name{刘凯文}

\medskip

\basicinfo{
  \faPhone ~ (+86) 153-0000-8888
  \textperiodcentered\
  \faEnvelope ~ username@email.com
  \textperiodcentered\
  \faGithub ~ github.com/username
}


% ----------------------------------------------------------------------------- %

\section{教育背景}
\dateditem{\textbf{上海交通大学} \quad 电子与计算机工程学院 \quad 本科}{2014年 -- 2018年}


% ----------------------------------------------------------------------------- %

\section{专业技能}
\smallskip
Golang、Linux、Mysql、ETL


% ----------------------------------------------------------------------------- %

% ----------------------------------------------------------------------------- %

\section{项目经历}


% ---
\datedproject{分布式流体计算项目}{竞赛项目}{2016年05月 -- 2016年06月}
\textit{性能优化、并行计算}
\vspace{0.4ex}

将一个单机的流体计算程序移植到多机平台
\begin{itemize}
  \item 计算节点内部使用OpenMP并行化
  \item 计算节点间使用MPI并行化
  \item 单机优化性能达2x, 多机接近线性加速比
\end{itemize}

% ---
\datedproject{这是一个测试项目}{导师项目}{2016年02月 -- 2016年04月}
\textit{编译技术、C++}
\vspace{0.4ex}

这是一个没有分点陈述的项目,如果没有分点陈述,LaTeX的排版会是什么样子?
这是一个没有分点陈述的项目,如果没有分点陈述,LaTeX的排版会是什么样子?
这是一个没有分点陈述的项目,如果没有分点陈述,LaTeX的排版会是什么样子?
这是一个没有分点陈述的项目,如果没有分点陈述,LaTeX的排版会是什么样子?

% ---
\datedproject{C-Compiler}{个人项目}{2016年02月 -- 2016年04月}
\textit{编译技术、C++}
\vspace{0.4ex}

使用C++实现了一个简单C语言编译器,支持生成到目标代码。
\begin{itemize}
  \item 支持C11标准大部分标准
  \item 实现了类似lex/flex的词法解析器生成工具
  \item 实现了类似yacc/bison的语法解析器生成工具
  \item 后端部分采用窥孔优化
\end{itemize}


% ----------------------------------------------------------------------------- %

\section{课外}
\dateditem{\textbf{华中科技大学XX团队}}{2016年06月 -- 至今}

\end{document}
